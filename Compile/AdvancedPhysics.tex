

















\chapter{Thermodynamics and Statistical Mechanics}
\begin{itemize}
\item $PV$ diagram plots change in pressure wrt to volume for some process.
The work done by the gas is the area \textbf{under the curve.}
\item If the cyclic process moves clockwise around the loop, then $W$ will
be positive, and it represents a heat engine. If it moves counterclockwise,
then $W$ will be negative, and it represents a heat pump.
\item The most basic definition of entropy is \[
dS\geq\frac{dQ}{T}\]

\item Heat transfer

\begin{itemize}
\item Conduction: rate\[
H=\frac{\Delta Q}{\Delta t}=-kA\frac{T_{2}-T_{1}}{L},\qquad\frac{dQ}{dt}=-kA\frac{dT}{dx}\]
where $A$ is area, $k$ is a constant.
\item Convection (probably not in GRE),\[
H=\frac{\Delta Q}{\Delta T}=hA(T_{s}-T_{\infty})\]
where $T_{s}$ is the surface temperature, $h=$convective heat-transfer
coefficient. There are both natural and forced convections.
\end{itemize}
\item Radiation\[
\text{Power}=\epsilon\sigma AT^{4}\]
$\epsilon=$emissivity, $\epsilon\in[0,1]$. Net loss$=\epsilon\sigma A(T_{\text{emission}}^{4}-T_{\text{absorption}}^{4})$
\item Wien's displacement law: The absolute temperature of a blackbody and
the peak wavelength of its radiation are inversely proportional:\[
\lambda_{\text{max}}T=2.898\times10^{-3}\ \text{m\ensuremath{\cdot}K}\]

\item Ideal gas law\[
PV=nRT=NkT\]

\item Kinetic theory of gas\[
P=\frac{1}{3}\rho v_{\text{rms}}^{2}\qquad v_{\text{rms}}=\sqrt{\frac{3kT}{m}},\qquad\bar{v}=\sqrt{\frac{8kT}{\pi m}},\qquad v_{\text{most probable}}=v_{m}=\sqrt{\frac{2kT}{m}}\]

\item Maxwell-Boltzmann distribution (less likely to be in GRE), number
of molecules with energy between $E$ and $E+dE$\[
N(E)dE=\frac{2N}{\sqrt{\pi}(kT)^{3/2}}\sqrt{E}e^{-E/kT}dE\]
\[
f(v)d^{3}v=\left(\frac{m}{2\pi kT}\right)^{3/2}e^{-mv^{2}/2kT}d^{3}v\]
\[
P(v)=\sqrt{\frac{2}{\pi}}\left(\frac{m}{kT}\right)^{3/2}v^{2}e^{-mv^{2}/2kT}\]
(from which we can derive $v_{m}$)
\item Mean free path of a gas molecule of radius $b$\[
l=\frac{1}{4\pi Tb^{2}(N/V)}\]

\item Van der Waals equation of state\[
(P+an^{2}/V^{2})(V-bn)=nRT\]
\[
(P+aN^{2}/V^{2})(V-Nb)=NkT\]

\item Adiabatic process\[
PV^{\gamma}=\text{const}\]
For an ideal gas to expand adiabatically from $(P_{1},V_{1})\to(P_{2},V_{2})$,
work done by the gas is\[
W=\frac{P_{1}V_{1}-P_{2}V_{2}}{\gamma-1}\]
derived from $W=\int_{V_{1}}^{V_{2}}PdV$.
\item The greatest possible thermal efficiency of an engine operating between
two heat reservoirs is that of a Carnot engine, one that operates
in the Carnot cycle. Max efficiency is\[
y^{\star}=1-\frac{T_{\text{cold}}}{T_{\text{hot}}}\]
For the case of the refrigerator\[
\kappa=\frac{Q_{\text{cold}}}{W}\qquad\kappa_{\text{Carnot}}=\left(\frac{T_{\text{hot}}}{T_{\text{cold}}}-1\right)^{-1}\]
Carnot=adiabatic+isothermal, $dS=0$. Otto=adiabatic+isobaric\[
y=1-\frac{T_{d}-T_{a}}{T_{c}-T_{b}}\]

\item Dalton's Law\[
P=P_{1}+P_{2}=(n_{1}+n_{2})\frac{RT}{V}\]

\item The critical isotherm is the line that just touches the critical liquid-vapor
region\[
\left(\frac{dP}{dV}\right)_{c}=0\qquad\left(\frac{d^{2}P}{dV^{2}}\right)_{c}=0\]
with $c$ the critical point. Equilibrium region is where pressure
and chemical potential for the two states of matter equal, usually
a pressure constant region in the $P-V$ diagram.
\item In the Dulong-Petit law,\[
C_{V}=\frac{dE}{dT}=3R\]

\item Laws of thermodynamics

\begin{itemize}
\item 0th: If two thermodynamic systems are each in thermal equilibrium
with a third, then they are in thermal equilibrium with each other.
\item 1st: $\Delta U=Q-W$ (conservation of energy)
\item 2nd: Entropy increases/heat flows from hot to cold/heat cannot be
completely converted into work.
\item 3rd: As $T\to0,$ $S\to$constant minimum.
\end{itemize}
\item Change in entropy for a system where specific heat and temperature
are constant;\[
\Delta S=Nk\ln\frac{V}{V_{0}}\]

\item Change in energy for an ideal gas:\[
\Delta U=C_{V}\Delta T\]

\item Work done by ideal gas:\[
W=\int PdV=\begin{cases}
\begin{array}{c}
NkT\ln\frac{V_{2}}{V_{1}}\\
P\Delta V\end{array} & \begin{array}{c}
\text{Isothermal}\\
\text{Ideal gas, constant Pressure}\end{array}\end{cases}\]

\item Partition function:\[
Z=\sum_{i}e^{-\beta E_{i}}=\int dE\:\Omega(E)e^{-\beta E}=\int dE\, e^{-\beta A(E)}\]
where $A(E)$ is the Helmholtz free energy and $\Omega(E)$ is the
degeneracy.\[
P(E_{i})=\frac{e^{-\beta E_{i}}}{Z}\]
\[
S=k\ln\Omega=-k\sum_{i}P_{i}\ln P_{i}\]

\item Equipartition Theorem: (1) Classical canonical and (2) quadratic dependence:
each particle has energy $\frac{1}{2}kT$ for each quadratic canonical
degree of freedom.
\item Internal energy\[
dU=TdS-PdV\]
Enthalpy\[
H=U+PV\qquad dH=TdS+VdP\qquad\text{isobaric}\]
Helmholtz\[
F=U-TS,\qquad dF=-SdT-PdV\qquad\text{isothermal}\]
Gibbs free energy\[
G=U-TS+PV,\qquad dG=-SdT+VdP\]

\item Heat capacities:\[
C_{V}=\left(\frac{\partial U}{\partial T}\right)_{V}=T\left(\frac{\partial S}{\partial T}\right)_{V}\]
\[
C_{P}=\left(\frac{\partial U}{\partial T}\right)_{P}+P\left(\frac{\partial V}{\partial T}\right)_{P}=T\left(\frac{\partial S}{\partial T}\right)_{P}=\left(\frac{\partial H}{\partial T}\right)_{P}\]

\item Fun stuff:\[
\langle E\rangle=-\frac{\partial}{\partial\beta}\ln Z,\qquad F=-kT\ln Z\]
\[
S=k\ln Z+\langle E\rangle/T,\qquad dS=\int\frac{dQ}{T}\]
Gibbs-Helmholtz equation.\[
U=F+TS=F-T\left(\frac{\partial F}{\partial T}\right)_{V}=-T^{2}\left(\frac{\partial}{\partial T}\right)_{V}\left(\frac{F}{T}\right)\]

\item Availability of system\[
A=U+P_{0}V-T_{0}S\]
In natural change, $A$ cannot increase.
\item Diatomic gas\[
U=\frac{5}{2}kT\]

\item Maxwell Relations\[
\left(\frac{\partial T}{\partial V}\right)_{S}=-\left(\frac{\partial P}{\partial S}\right)_{V}=\frac{\partial^{2}U}{\partial S\partial V}\]
\[
\left(\frac{\partial T}{\partial P}\right)_{S}=\left(\frac{\partial V}{\partial S}\right)_{P}=\frac{\partial^{2}H}{\partial S\partial P}\]
\[
\left(\frac{\partial S}{\partial V}\right)_{T}=\left(\frac{\partial P}{\partial T}\right)_{V}=-\frac{\partial^{2}A}{\partial T\partial V}\]
\[
-\left(\frac{\partial S}{\partial P}\right)_{T}=\left(\frac{\partial V}{\partial T}\right)_{P}=\frac{\partial^{2}G}{\partial T\partial P}\]

\item For ideal gas in adiabatic process, $W=\Delta U=\frac{3}{2}Nk\Delta T$
\item Clockwise enclosed area in a $P-V$ diagram is the work done \underbar{by
}the gas in a cycle.
\item Chemical potential\[
\mu(T,V,N)=\left(\frac{\partial F}{\partial N}\right)_{T,V}\]
At equilibrium $\mu$ is uniform, $F$ achieves minimum.
\item $P_{\text{boson}}\propto T^{5/2}$, $P_{\text{classical}}\propto T$,
$P_{\text{fermion}}\propto T_{F}$ (very big). $T_{\text{classical}}\gg T_{\text{boson}}$
\item A thermodynamic system in maximal probability state is stable.
\item Both Debye and Einstein assume $3N$ independent Harmonic oscillators
for lattice. Einstein took a constant frequency
\end{itemize}
















\chapter{Quantum Mechanics}
\begin{itemize}
\item Uncertainty principle\[
\Delta x\Delta p\geq\frac{\hbar}{2},\qquad\Delta E\Delta t\geq\frac{\hbar}{2}\]

\item Schrodinger equation\[
i\hbar\frac{\partial\Psi}{\partial t}=-\frac{\hbar^{2}}{2m}\nabla^{2}\Psi+V\Psi\]

\item Commutator relation:\[
[AB,C]=ABC-CAB=ABC-ACB+ACB-CAB=A[B,C]+[A,C]B\]

\item De Broglie\[
\lambda=\frac{h}{p}=\frac{hc}{E}=\frac{h}{\sqrt{2mkT}}\]
(The last equality is thermal)
\item A one-dimensional problem has no degenerate states.
\item Heisenberg's uncertainty principle generalized:\[
\Delta A\Delta B\geq\frac{1}{2}|\langle[A,B]\rangle|\]

\item Infinite square well\[
\psi_{n}=\sqrt{\frac{2}{a}}\sin\frac{n\pi x}{a},\qquad E_{n}=\frac{n^{2}\pi^{2}\hbar^{2}}{2ma^{2}},\qquad n\geq1\]
Delta-function well $V=-\alpha\delta(x)$. Only one bound state, many
scattering states.\[
\psi(x)=\sqrt{\frac{m\alpha}{\hbar}}e^{-m\alpha|x|/\pi^{2}},\qquad E=-\frac{m\alpha^{2}}{2\hbar^{2}}\]
Shallow, narrow well, there is always at least one bound state.
\item Selection rule\[
\Delta l=\pm1,\qquad\Delta m_{l}=\pm1\text{ or}\ 0,\qquad\Delta j=\pm1\ \text{or }0\]
Electric dipole radiation $\Leftrightarrow\Delta l=0$. Magnetic dipole
or electric quadrupole transitions are {}``forbidden'' but do occur
occasionally.
\item Stimulated and spontaneous emission rate $\propto|p|^{2}$ where\[
p\equiv q\langle\psi_{b}|z|\psi_{a}\rangle\]
The lifetime of an excited state is $\tau=\left(\sum A_{i}\right)^{-1}$
where $A_{i}$ are spontaneous emission rates.
\item Time-independent first order perturbation\[
E_{n}^{1}=E_{n}^{0}+\langle\psi_{n}^{0}|H'|\psi_{n}^{0}\rangle,\qquad\psi_{n}^{1}=\psi_{n}^{0}+\sum_{m\neq n}\frac{\langle\psi_{m}^{0}|H'|\psi_{n}^{0}\rangle}{E_{n}^{0}-E_{m}^{0}}\psi_{m}^{0}\]

\item Quantum approximation of rotational energy\[
E_{\text{rot}}=\frac{\hbar^{2}l(l+1)}{2I}\]

\item Fermi energy\[
E_{F}=kT_{F}\simeq\frac{1}{2}mv^{2}\]

\item Differential cross-section\[
\frac{d\sigma}{d\Omega}=\frac{\text{scattered flux/unit of solid angle}}{\text{incident flux/unit of surface}}\]

\item Intrinsic magnetic moment \[
\vec{\mu}=\gamma\vec{S},\qquad\gamma=\frac{eg}{2m}\]
where $g$ is the Lande $g$-factor. If $m$ points up, $\vec{\mu}$points
down.
\item Total cross section\[
\sigma=\int D(\theta)d\Omega,\qquad D(\theta)=\frac{d\sigma}{d\Omega}\]

\item Stark effect is the electrical analog to the Zeeman effect.
\item Born-Oppenheimer approximation: the assumption that the electronic
motion and the nuclear motion in molecules can be separated, i.e.\[
\psi_{\text{molecule}}=\psi_{e}\psi_{\text{nuclei}}\]

\item In Stern-Gerlach experiment, a beam of neutral silver atoms are sent
through an inhomogeneous magnetic field. Classically, nothing happens
as the atoms are neutral with Larmor precession, the beam would be
deflected into a smear. But it actually deflects into $2s+1$ beams,
thus corroborating with the fact electrons are at spin $\frac{1}{2}$
\item Know the basic spherical harmonics\[
Y_{0}^{0}=\sqrt{\frac{1}{4\pi}},\qquad Y_{1}^{\pm1}=\mp\sqrt{\frac{3}{8\pi}}\sin\theta e^{\pm i\phi},\qquad Y_{1}^{0}=\sqrt{\frac{3}{4\pi}}\cos\theta\]

\item Probability density current\[
\vec{J}=\frac{\hbar}{2mi}(\psi^{*}\nabla\psi-\psi\nabla\psi^{*})=\Re\left(\psi^{*}\frac{\hbar}{im}\nabla\psi\right)\]

\item Laser operates by going from lower state to high state (population
inversion), then falls back on a metastable state in between (not
all the way down due to selection rule).
\item Neat identities:\[
\langle\mathcal{O}\rangle=\int\Psi^{*}\mathcal{O}\Psi dx,\qquad[f(x),p]=i\hbar\frac{\partial f}{\partial x},\qquad p=-i\hbar\nabla\]

\item Ehrenfest's Theorem: expectation values obey classical laws.\[
m\frac{d^{2}\langle x\rangle}{dt^{2}}=\frac{d\langle p\rangle}{dt}=\left\langle -\frac{\partial V}{\partial x}\right\rangle \]

\item If $V(x)$ is even, $\psi(x)$ can always be taken to be even or odd.
\item More identities:\[
\langle H\rangle=\sum_{n=1}^{\infty}|c_{n}|^{2}E_{n},\qquad\delta(x)=\frac{1}{2\pi}\int_{-\infty}^{\infty}e^{ikx}dk\]

\item Tunneling shows exponential decay.
\item The ground state of even potential is even and has no nodes.
\item In stationary states, all expectation values are independent of $t$.
\item Harmonic oscillators:\[
H=\hbar\omega(a_{-}a_{+}-\tfrac{1}{2})=\hbar\omega(a_{+}a_{-}+\tfrac{1}{2}),\qquad a_{\pm}=\frac{1}{\sqrt{2\hbar m\omega}}(m\omega x\mp ip)\]
\[
[a_{-}a_{+}]=1,\qquad N\equiv a_{+}a_{-},\qquad N\psi_{n}=n\psi_{n}\]
\[
a_{+}\psi_{n}=\sqrt{n+1}\psi_{n+1},\qquad a_{-}\psi_{n}=\sqrt{n}\psi_{n-1}\]
\[
\psi_{n}=\frac{1}{\sqrt{n!}}(a_{+})^{n}\psi_{0},\qquad x=\sqrt{\frac{\hbar}{2m\omega}}(a_{+}+a_{-}),\qquad p=i\sqrt{\frac{\hbar m\omega}{2}}(a_{+}-a_{-})\]

\item Fourier transforms:\[
\Phi(p,t)=\frac{1}{\sqrt{2\pi\hbar}}\int_{-\infty}^{\infty}e^{-px/\hbar}\Psi(x,t)dx\]
\[
\Psi(x,t)=\frac{1}{\sqrt{2\pi\hbar}}\int_{-\infty}^{\infty}e^{ipx/\hbar}\Phi(p,t)dp\]

\item Operators changing in time:\[
\frac{d\langle Q\rangle}{dt}=\frac{i}{\hbar}\langle[H,Q]\rangle+\left\langle \frac{\partial Q}{\partial t}\right\rangle \]

\item Virial theorem, in stationary state\[
2\langle T\rangle=\left\langle x\frac{dV}{dx}\right\rangle \]

\item Hydrogen atom revisited:\begin{align*}
E_{n} & \propto\text{reduced mass}\\
 & \propto Z^{2}\\
 & \propto1/n^{2}\\
 & =-\left[\frac{m}{2\hbar^{2}}\left(\frac{e^{2}}{2\pi\epsilon_{0}}\right)^{2}\right]\frac{1}{n^{2}}=\frac{E_{1}}{n^{2}}\\
E_{n}(Z) & =Z^{2}E_{n}\\
a(Z) & =\frac{a}{Z}\\
R(Z) & =Z^{2}R\end{align*}
Bohr radius $a=4\pi\epsilon_{0}\hbar^{2}/me^{2}=0.528\times10^{-10}$
meters.\[
\psi_{100}(r,\theta,\phi)=\frac{1}{\sqrt{\pi a^{3}}}e^{-r/a}\]

\item Angular momentum\[
[L_{i},L_{j}]=i\hbar L_{k}\epsilon_{ijk}\]
where $\epsilon_{ijk}=1$ for even permutations, -1 for odd permutations,
zero otherwise.\[
L_{\pm}=L_{x}\pm iL_{y},\qquad[L^{2},L_{i}]=0\]
\[
L^{2}f_{l}^{m}=\hbar^{2}l(l+1),\qquad L_{z}f_{l}^{m}=\hbar mf_{l}^{m}\]
\[
L_{\pm}f_{l}^{m}=\hbar\sqrt{(l\mp m)(l\pm m+1)}f_{l}^{m}=\hbar\sqrt{l(l+1)-m(m\pm1)}f_{l}^{m}\]
\[
[L_{z},x]=i\hbar y,\qquad[L_{z},p_{x}]=i\hbar p_{y},\qquad[L_{z},y]=-i\hbar x,\qquad[L_{z},p_{y}]=-i\hbar p_{x}\]
\[
L_{z}=\frac{\hbar}{i}\frac{\partial}{\partial\phi}\]

\item Spin, \[
S^{2}=\frac{3}{4}\hbar^{2}\left(\begin{array}{cc}
1 & 0\\
0 & 1\end{array}\right)\]
\[
\sigma_{x}=\left(\begin{array}{cc}
0 & 1\\
1 & 0\end{array}\right),\qquad\sigma_{y}=\left(\begin{array}{cc}
0 & -1\\
1 & 0\end{array}\right),\qquad\sigma_{z}=\left(\begin{array}{cc}
1 & 0\\
0 & -1\end{array}\right)\]
\[
\vec{S}=\frac{\hbar}{2}\vec{\sigma}\]
\[
\chi_{+}^{(x)}=\left(\begin{array}{c}
1/\sqrt{2}\\
1/\sqrt{2}\end{array}\right),\qquad\chi_{-}^{(x)}=\left(\begin{array}{c}
1/\sqrt{2}\\
-1/\sqrt{2}\end{array}\right)\]
\[
\langle S_{x}^{2}\rangle=\langle S_{y}^{2}\rangle=\langle S_{z}^{2}\rangle=\frac{\hbar^{2}}{4}\]

\item Clebsch-Gorden coefficients\[
|sm\rangle=\sum_{m_{1}+m_{2}=m}C_{m_{1}m_{2}m}^{s_{1}s_{2}s}|s_{1}m_{1}\rangle|s_{2}m_{2}\rangle\]
\[
|s_{1}m_{1}\rangle|s_{2}m_{2}\rangle=\sum_{s}C_{m_{1}m_{2}m}^{s_{1}s_{2}s}|sm\rangle\]

\item Continuity equation\[
\nabla\cdot\vec{J}=-\frac{\partial}{\partial t}|\psi|^{2}\]
\[
\int_{S}\vec{J}\cdot d\vec{a}=-\frac{d}{dt}\int_{V}|\psi|^{2}d^{3}\vec{r}\]

\item Representation of angular momentum.\[
^{2s+1}\mathcal{L}_{J}\]
where $s=$spin, $\mathcal{L}=$orbital, $J=$total. Hund's rule:
(1) State with highest spin will have lowest energy given Pauli principle
satisfied; (2) For given spin and anti-symmetrization highest $\mathcal{L}$
have lowest energy; (3) Lowest level has $J=|L-S|,$ if more than
half-filled $J=L+S$.
\item Fermi gas\[
k_{F}=(3\rho\pi^{2})^{1/3},\qquad\rho=Nq/V,\qquad v_{F}=\sqrt{2E_{F}/m}\]
Degeneracy pressure\[
P\propto\rho^{5/3}m_{e}^{-1}m_{p}^{-5/3}\]

\item Particle distributions\[
n(\epsilon)=\begin{cases}
\begin{array}{c}
e^{-\beta(\epsilon-\mu)}\\
(e^{\beta(\epsilon-\mu)}+1)^{-1}\\
(e^{\beta(\epsilon-\mu)}-1)^{-1}\end{array} & \begin{array}{c}
\text{Maxwell-Boltzmann}\\
\text{Fermi-Dirac}\\
\text{Bose-Einstein}\end{array}\end{cases}\]
Blackbody density\[
\rho(\omega)=\frac{\hbar\omega^{3}}{\pi^{2}c^{3}(e^{\hbar\omega/kT}-1)}\]

\item Fine structure$\to$spin-orbit coupling. Relativistic correction $\alpha=1/137.056$.
Then Lamb shift is from the electric field, then Hyperfine structure
due to magnetic interaction between electrons and protons, then spin-spin
coupling (21 cm line)
\item Fine structure breaks degeneracy in $l$ but still have $j$
\item Fermi's golden rule is a way to calculate the transition rate (probability
of transition per unit time) from one energy eigenstate of a quantum
system into a continuum of energy eigenstates, due to a perturbation.
\item Full shell and close to a full shell configuration are more difficult
to ionize.
\item Larmor precession:\[
\vec{\Gamma}=\vec{\mu}\times\vec{B}=\gamma\vec{J}\times\vec{B}\]
and we get $\omega=\gamma B$, where $\Gamma$ is the torque, $\mu$
is the magnetic moment, and $J$ is total angular momentum.
\end{itemize}









\chapter{Atomic Physics}
\begin{itemize}
\item $\Delta E=hf=\hbar\omega=hc/\lambda$. $hc=12.4\ \text{keV\ensuremath{\cdot}\AA}=1240\ \text{eV\ensuremath{\cdot}nm}$,
de Broglie wavelength $\lambda=h/p$.
\item Emission due to transition from level $n$ to level $m$\[
\frac{1}{\lambda}=R\left(\frac{1}{m^{2}}-\frac{1}{n^{2}}\right)\]
$m=1$ Lyman series, $m=2$ Balmer series.\[
R=1.097\times10^{7}\text{m}^{-1},\qquad E_{n}=-\frac{13.6\ \text{eV}}{n^{2}}\]

\item Hydrogen model extended, $Z=$number of protons, quantities scale
as \[
E\sim Z^{2},\qquad\lambda\sim\frac{1}{Z^{2}}\]
Reduced-mass correction to emission formula is \[
\frac{1}{\lambda}=\frac{RZ^{2}}{1+m/M}\left(\frac{1}{n_{f}^{2}}-\frac{1}{n_{i}^{2}}\right)\]
where $m$ is the mass of electron, $M$ is the mass of the proton,
$m/M=1/1836$.
\item Bohr postulate $L=mvr=n\hbar$
\item Zeeman effect: splitting of a spectral line into several components
in the presence of a static magnetic field.
\item $k$ series refers to the innermost shell ($K$, $L$, $M$, $N$)
so transition to innermost shell.\[
E=-13.6(Z-1)^{2}\left(1-\frac{1}{n_{i}^{2}}\right)\ \text{eV}\]
where the $(Z-1)^{2}$ is a shielding approximation.
\item Frank-Hertz Experiment: Electrons of a certain energy range can be
scattered inelastically, and the energy lost by electrons is discrete.
\item Spectroscopic notation is a standard way to write down the angular
momentum quantum number of a state, \[
^{2s+1}L_{j}\]
where $s$ is the total spin quantum number, $2s+1$ is the number
of spin states, $L$ refers to the orbital angular momentum quantum
number $\ell$ but is written as $S,P,D,F,\ldots$ for $\ell=0,1,2,3,\ldots$
and $j$ is the total angular momentum quantum number. So for hydrogen
we could have things like\[
^{2}P_{\frac{3}{2}},^{2}P_{\frac{1}{2}}\]
(since $s=1/2$ and $\ell=1$, spin up versus spin down).
\end{itemize}












\chapter{Special Relativity}
\begin{itemize}
\item Energy:\[
E^{2}=(pc)^{2}+(mc^{2})^{2}\]
For massless particles, $E=pc=h\nu$
\item Relativistic Doppler Effect\[
\lambda=\sqrt{\frac{1\pm\beta}{1\mp\beta}}\lambda_{0}\]
$\beta=v/c$. Sign is determined by whether source is moving away
or closer.
\item Space-time interval\[
\Delta s^{2}=c^{2}\Delta t^{2}-\Delta x^{2}-\Delta y^{2}-\Delta z^{2}\]

\item Lorentz transformation\[
\left(\begin{array}{c}
ct'\\
x'\\
y'\\
z'\end{array}\right)=\left(\begin{array}{cccc}
\gamma & -\beta\gamma & 0 & 0\\
-\beta\gamma & \gamma & 0 & 0\\
0 & 0 & 1 & 0\\
0 & 0 & 0 & 1\end{array}\right)\left(\begin{array}{c}
ct\\
x\\
y\\
z\end{array}\right)\]

\item Relativistic addition of velocities\[
u_{x}'=\frac{u_{x}+v}{1+u_{x}v/c^{2}},\qquad u_{y}'=\frac{u_{y}}{\gamma(1+u_{x}v/c^{2})},\qquad u_{z}'=\frac{u_{z}}{\gamma(1+u_{x}v/c^{2})},\qquad\gamma\equiv\frac{1}{\sqrt{1-\beta^{2}}}\]

\item Lorentz-Transformation of EM, parallel and perpendicular to direction
o motion.\[
\vec{E}_{\parallel}'=\vec{E}_{\parallel},\qquad\vec{E}'_{\perp}=\gamma(\vec{E}_{\perp}+\vec{v}\times\vec{B}_{\perp})\]
\[
\vec{B}'_{\parallel}=\vec{B}_{\parallel},\qquad\vec{B}'_{\perp}=\gamma(\vec{B}_{\perp}-\vec{v}\times\vec{E}_{\perp}/c^{2})\]

\item Relativistic energy/momentum\[
E=\gamma mc^{2},\qquad p=\gamma mv\]

\item In every closed system, the total relativistic energy and momentum
are conserved.
\item Spacelike separation means two events can happen at the same time,
which requires \[
\Delta s^{2}=c^{2}\Delta t^{2}-\Delta x^{2}<0\]

\item Transverse Doppler shift:\[
f=\frac{f'}{\sqrt{1-\beta^{2}}}\text{ or }f=f'\sqrt{1-\beta^{2}}\]

\item Four-vectors can be useful. We can define\[
\mathbf{P}=\left(\frac{E}{c},\mathbf{p}\right)\]
and the dot product\[
\mathbf{P}^{2}=\frac{E^{2}}{c^{2}}-p^{2}=m^{2}c^{2}\]
to get \[
E^{2}=m^{2}+p^{2}.\]
Remember, this mass is invariant, so we can equate the $\mathbf{P}$
vector at different times.
\end{itemize}















\chapter{Laboratory Methods}
\begin{itemize}
\item If measurements are independent (or intervals in a Poisson process
are independent) both expected value and variance increase linearly
with time, so longer time can improve uncertainty, which is usually
defined as \[
\frac{\sigma}{R}\propto\frac{1}{\sqrt{t}}\]

\item In Poisson distribution, $\sigma=\sqrt{\bar{x}}$.
\item Error analysis, estimating uncertainties. If you are sure the value
is closer to 26 than to 25 or 27, then record best estimate $26\pm0.5$.
\item Propagation of uncertainties for sum of random and independent variables\[
\delta x=\sqrt{\sum_{i}(\delta x_{i})^{2}}\]
If multiplication or divisions are involved, use fractional uncertainty:\[
\frac{\delta q}{|q|}=\sqrt{\sum_{i}\left(\frac{\delta x_{i}}{x_{i}}\right)^{2}}\]

\item Experimental uncertainties can be revealed by repeating the measurements
are called random errors; those that cannot be revealed in this way
are called systematic errors.
\item If the the uncertainties are different for different measurements,
we have\[
\bar{x}=\frac{\sum(x_{i}/\sigma_{i}^{2})}{\sum_{i}(1/\sigma_{i})^{2}}\qquad\sigma_{\bar{x}}^{2}=\frac{1}{\sum_{i}(1/\sigma_{i}^{2})}\]

\end{itemize}







\chapter{Specialized Topics}
\begin{itemize}
\item Photoelectric effect.\[
E_{\text{photon}}=\phi+K_{\text{max}}\]
(or the sum of the work function and the kinetic energy).
\item Compton scattering: \[
\lambda'-\lambda=\frac{h}{m_{e}c}(1-\cos\theta)\]
where $m_{e}$ is the mass of the atom: $h/m_{e}c$ is the Compton
wavelength of the electron, and $\lambda'$ is the new wavelength.
\item X-ray Bragg reflection \[
n\lambda=2d\sin\theta\]
(compare to \textbf{diffraction grating} $n\lambda=d\sin\theta$)
\item $1.602\times10^{-19}$J$=e(1\ \text{V})=1$ eV.
\item In solid-state physics, effective mass is \[
m^{*}=\frac{\hbar^{2}}{d^{2}E/dk^{2}}\]

\item Electronic filters: high pass means $\omega\to\infty$, $V_{\text{in}}=V_{\text{out}}$.
Usually look at $I=V_{\text{in}}/Z$, $Z=R+i(X_{L}-X_{C})$, $X_{L}=\omega L,$
$X_{C}=1/\omega C$.
\item Band spectra is a term that refers to using EM waves to probe molecules.
\item Solid state: \[
\text{primitive cell}=\frac{\text{unit cell}}{\text{\# of lattice points in a Bravais lattice}}\]
Simple cubic $\to$ 1 point, body-centered $\to$ 2 points, face-centered
$\to$ 4 points.
\item Resistivity of undoped semiconductor varies as $1/T$.
\item Nuclear physics: binding energy is a form of potential energy, convention
is to take it as positive. It's the energy needed to separate into
different constituents. It is usually subtracted for other energy
to tally total energy.
\item Pair production refers to the creation of an elementary particle and
its antiparticle. Usually need high energy (at least the total mass).
\item At low energies, photoelectric-effect dominates Compton scattering.
\item Radioactivity: Beta decay\[
X_{Z}^{A}\to X_{Z+1}^{'A}+\beta_{-1}^{0}+\nu\]
Alpha:\[
X_{Z}^{A}\to X_{Z-2}^{\prime A-4}+\text{He}_{2}^{4}\]
Gamma\[
X_{Z}^{A}\to X_{Z}^{A}+\gamma\]
Deuteron decay (not natural)\[
X_{Z}^{A}\to X_{Z-1}^{A-2}+\text{H}_{1}^{2}\]
Radioactivity usually follows Poisson distribution.
\item Coaxial cable terminated at an end with characteristic impedance in
order to avoid reflection of signals from the terminated end of cable.
\item Human eyes can only see things in motion up to $\sim25$ Hz.
\item In magnetic field, $e$ are more likely to be emitted in a direction
opposite to the spin direction of the decaying atom.
\item Op-amp (operational amplifiers): if you only have two days to prepare
for the GRE, this is not worth the effort, maximum one question on
this. Read {}``The Art of Electronics'' to check this out.
\item The specific heat of a superconductor jumps to a lower value at the
critical temperature (resistivity jumps too)
\item Elementary particles: review the quarks, leptons, force carriers,
generations, hadrons.

\begin{itemize}
\item Family number conserved
\item Lepton number conserved
\item Strangeness is conserved (except for weak interactions)
\item Baryon number is conserved
\end{itemize}
\item Internal conversion is a radioactive decay where an excited nucleus
interacts with an electron in one of the lower electron shells, causing
the electron to be emitted from the atom. It is not beta decay.
\end{itemize}











