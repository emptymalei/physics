% !TEX program = xelatex

%This document had would had been used on Ways to Singularity, which is a website that supports MathJax. So some html elements may occur in this document. DELETE THIS when publishing. (If you are not very clear on the grammar I used here, read the academic publication called Time Traveller's Handbook of 1001 Tense Formations by Dr Dan Streetmentioner, which had would had been publish in year 220010 of Gregorian Calendar.)

\documentclass[12pt,a4paper]{book}

\usepackage{amsthm,amsfonts,amssymb,bm}
\usepackage[fleqn]{amsmath}

% Page settings
\addtolength{\textheight}{2.0cm}
\addtolength{\voffset}{-2cm}
\addtolength{\hoffset}{-1.0cm}
\addtolength{\textwidth}{2.0cm}

%\allowdisplaybreaks

%\usepackage{subeqnarray}
\usepackage{mathrsfs}
%\usepackage{color}
%\usepackage{url}
%\usepackage{ulem}
\usepackage{indentfirst}   % Indent first line of a paragraph
%\usepackage{textcomp}

\usepackage{enumerate}


%%Here is the configuration for chinese. setmainfont is the default font of the text.
\usepackage[cm-default]{fontspec}
\usepackage{xunicode}
%\usepackage{xltxtra}
\setmainfont{Arial}
%\setsansfont[BoldFont=Arial]{KaiTi_GB2312}
%\setmonofont{NSimSun}




%\XeTeXlinebreaklocale "zh"
%\XeTeXlinebreakskip = 0pt plus 1pt

% Figure, Diagram, Caption settings
%\usepackage{tikz}
%\usetikzlibrary{mindmap,trees}
\usepackage{graphicx}
%\usepackage{graphics}
%\usepackage[hang,small,bf]{caption}
%\setlength{\captionmargin}{50pt}

% Redefine some fonts.
%\newfontfamily\heiti{"黑体"}
%\newfontfamily\fs{"仿宋"}
%\newfontfamily\yahei{"微软雅黑"}


%includeonly{}


\graphicspath{{Figures/}{Figures/Chap1/}{Figures/Chap2/}{Figures/Chap3/}}


\begin{document}
\title{Research Survival Handbook \\ (\textbf{Unfinished})}
\author{{\bf MA} Lei  \\
@ Interplanetary Immigration Agency \\
{\small\em \copyright \ Draft date \today}}
\date{}
%\begin{document}
\maketitle

% Redefine some math commands and environments.

\newcommand{\dd}{\mathrm d}
%\newcommand{\HH}{\mathcal H}
%\newcommand{\CN}{{\it Cosmologia Notebook}}
\newenvironment{eqnset}
{\begin{equation}\left \bracevert \begin{array}{l}}
{\end{array} \right. \end{equation}}

\newenvironment{eqn}
{\begin{equation}\left \bracevert \begin{array}{l}}
{\end{array} \right. \end{equation}}


























\section{How to survive the calculations of Special Relativity}

\subsection{Important Relations}
\begin{quotation}
Metric in use
\begin{equation}\eta_{\mu\nu}=\left(\begin{matrix}
	-1 & 0 & 0 & 0\\
	0 & 1 & 0 & 0\\
	0 & 0 & 1 & 0\\
	0 & 0 & 0 & 1\\
\end{matrix}\right)\end{equation}
\end{quotation}




\section{Quantities and Operations}

\paragraph{d'Alembertian}
d'Alembert operator, or wave operator, is the Lapace operator in Minkowski space.
\footnote{Actually, there are more general definations for Lapacian, which includes this d'Alembertian of course.}
\begin{eqnarray}
\Box\equiv \partial_\mu\partial^\nu&=&\eta_{\mu\nu}\partial^\mu \partial^\nu
\end{eqnarray}

In the usual {t,x,y,z} natural orthonormal basis,
\begin{eqnarray}
 \Box&=&-\partial_t^2+\partial_x^2+\partial_y^2+\partial_z^2 \\
&=&-\partial_t^2+\Delta^2 \\
&=&-\partial_t^2+\nabla
\end{eqnarray}

\begin{quotation}
On wiki \footnote{wiki:D'Alembert\_operator}, they give some applications to it.
\paragraph{klein-Gordon equation} $(\Box+m^2)\phi=0$
\paragraph{wave equation for electromagnetic field in vacuum} For the electromagnetic four-potential $\Box A^\mu=0$\footnote{Gauge}
\paragraph{wave equation for small vibrations} $\Box_c u(t,x)=0\rightarrow u_{tt}-c^2 u_{xx}=0$
\end{quotation}




\section{Fields and Particles}

\subsection{Energy-Momentum Tensor for Particles}

\begin{equation}
S_p \equiv -m c \int \int \mathrm d s\mathrm d\tau \sqrt{-\dot x ^\mu g_{\mu\nu} \dot x^\nu} \delta^4(x^\mu - x^\mu (s))    ,
\end{equation}
in which $x^\mu(s)$ is the trajectory of the particle. Then the energy density $\rho$ corresponds to $m\delta^4(x^\mu- x^\mu(s))$.

The Largrange density
\begin{equation}
\mathcal L = -\int\mathrm ds mc \sqrt{-\dot x^\mu g_{\mu\nu}\dot x^\nu}\delta^4(x^\mu - x^\mu(s))
\end{equation}

Energy-momentum density is $\mathcal T^{\mu\nu} = \sqrt{-g}T^{\mu\nu}$ is
\begin{equation} 
\mathcal T^{\mu\nu} = -2 \frac{\partial \mathcal L}{\partial g_{\mu\nu}}
\end{equation}

Finally,
\begin{eqnarray}
\mathcal T^{\mu\nu} &=& \int \mathrm ds \frac{mc\dot x^\mu \dot x^\nu}{\sqrt{-\dot x^\mu g_{\mu\nu} \dot x^\nu}} \delta(t-t(s))\delta^3(\vec x - \vec x(t)) \\
&=& m\dot x^\mu \dot x^\nu \frac{\mathrm d s}{\mathrm d t} \delta^3(\vec x - \vec x(s(t)))
\end{eqnarray}






\section{Theorems}

\subsection{Killing Vector Related}

\newtheorem{theorem}{}[chapter]

\begin{theorem}[]
$\xi^a$ is Killing vector field, $T^a$ is the tangent vector of geodesic line. Then $T^a\nabla_a(T^b\xi_b)=0$, that is $T^b\xi_b$ is a constant on geodesics.
\end{theorem}






\section{Topics}
\subsection{Redshift}

In geometrical optics limit, the angular frequency $\omega$ of a photon with a 4-vector $K^a$, measured by a observer with a 4-velocity $Z^a$, is $\omega=-K_aZ^a$.

\subsection{Stationary vs Static}

\paragraph{Stationay}
"A stationary spacetime admits a timelike Killing vector field. That a stationary spacetime is one in which you can find a family of observers who observe no changes in the gravitational field (or sources such as matter or electromagnetic fields) over time."

When we say a field is stationary, we only mean the field is time-independent.

\paragraph{Static}
"A static spacetime is a stationary spacetime in which the timelike Killing vector field has vanishing vorticity, or equivalently (by the Frobenius theorem) is hypersurface orthogonal. A static spacetime is one which admits a slicing into spacelike hypersurfaces which are everywhere orthogonal to the world lines of our 'bored observers'"

When we say a field is static, the field is both time-independent and symmetric in a time reversal process.








\end{document}