% !TEX program = xelatex

%This document had would had been used on Ways to Singularity, which is a website that supports MathJax. So some html elements may occur in this document. DELETE THIS when publishing. (If you are not very clear on the grammar I used here, read the academic publication called Time Traveller's Handbook of 1001 Tense Formations by Dr Dan Streetmentioner, which had would had been publish in year 220010 of Gregorian Calendar.)

\documentclass[12pt,a4paper]{book}

\usepackage{amsthm,amsfonts,amssymb,bm}
\usepackage[fleqn]{amsmath}

% Page settings
\addtolength{\textheight}{2.0cm}
\addtolength{\voffset}{-2cm}
\addtolength{\hoffset}{-1.0cm}
\addtolength{\textwidth}{2.0cm}

%\allowdisplaybreaks

%\usepackage{subeqnarray}
\usepackage{mathrsfs}
%\usepackage{color}
%\usepackage{url}
%\usepackage{ulem}
\usepackage{indentfirst}   % Indent first line of a paragraph
%\usepackage{textcomp}

\usepackage{enumerate}


%%Here is the configuration for chinese. setmainfont is the default font of the text.
\usepackage[cm-default]{fontspec}
\usepackage{xunicode}
%\usepackage{xltxtra}
\setmainfont{Arial}
%\setsansfont[BoldFont=Arial]{KaiTi_GB2312}
%\setmonofont{NSimSun}




%\XeTeXlinebreaklocale "zh"
%\XeTeXlinebreakskip = 0pt plus 1pt

% Figure, Diagram, Caption settings
%\usepackage{tikz}
%\usetikzlibrary{mindmap,trees}
\usepackage{graphicx}
%\usepackage{graphics}
%\usepackage[hang,small,bf]{caption}
%\setlength{\captionmargin}{50pt}

% Redefine some fonts.
%\newfontfamily\heiti{"黑体"}
%\newfontfamily\fs{"仿宋"}
%\newfontfamily\yahei{"微软雅黑"}


%includeonly{}


\graphicspath{{Figures/}{Figures/Chap1/}{Figures/Chap2/}{Figures/Chap3/}}


\begin{document}
\title{Research Survival Handbook \\ (\textbf{Unfinished})}
\author{{\bf MA} Lei  \\
@ Interplanetary Immigration Agency \\
{\small\em \copyright \ Draft date \today}}
\date{}
%\begin{document}
\maketitle

% Redefine some math commands and environments.

\newcommand{\dd}{\mathrm d}
%\newcommand{\HH}{\mathcal H}
%\newcommand{\CN}{{\it Cosmologia Notebook}}
\newenvironment{eqnset}
{\begin{equation}\left \bracevert \begin{array}{l}}
{\end{array} \right. \end{equation}}

\newenvironment{eqn}
{\begin{equation}\left \bracevert \begin{array}{l}}
{\end{array} \right. \end{equation}}





















\chapter{Special Relativity}



%%%%%%%%%%%%%%%%%%%%%%%%%%%%%%%%%%%%%%%%%%%%%%%%


\section{Conventions}

Metric in special relativity
\begin{equation}\eta_{\mu\nu}=\left(\begin{matrix}
	-1 & 0 & 0 & 0\\
	0 & 1 & 0 & 0\\
	0 & 0 & 1 & 0\\
	0 & 0 & 0 & 1\\
\end{matrix}\right)\end{equation}





\section{Quantities and Operations}

\paragraph{d'Alembertian}
d'Alembert operator, or wave operator, is the Lapace operator in Minkowski space.
\footnote{Actually, there are more general definations for Lapacian, which includes this d'Alembertian of course.}
\begin{eqnarray}
\Box\equiv \partial_\mu\partial^\nu&=&\eta_{\mu\nu}\partial^\mu \partial^\nu
\end{eqnarray}

In the usual {t,x,y,z} natural orthonormal basis,
\begin{eqnarray}
 \Box&=&-\partial_t^2+\partial_x^2+\partial_y^2+\partial_z^2 \\
&=&-\partial_t^2+\Delta^2 \\
&=&-\partial_t^2+\nabla
\end{eqnarray}

\begin{quotation}
On wiki \footnote{wiki:D'Alembert\_operator}, they give some applications to it.
\paragraph{klein-Gordon equation} $(\Box+m^2)\phi=0$
\paragraph{wave equation for electromagnetic field in vacuum} For the electromagnetic four-potential $\Box A^\mu=0$\footnote{Gauge}
\paragraph{wave equation for small vibrations} $\Box_c u(t,x)=0\rightarrow u_{tt}-c^2 u_{xx}=0$
\end{quotation}







%%%%%%%%%%%%%%%%%%%%%%%%%%%%%%%%%%%%%%%%%%
%%%%%%%%%%%%%%%%%%%%%%%%%%%%%%%%%%%%%%%%%%
%%%%%%%%%%%%%%%%%%%%%%%%%%%%%%%%%%%%%%%%%%
%%%%%%%%%%%%%%%%%%%%%%%%%%%%%%%%%%%%%%%%%%







\chapter{General Relativity}






%%%%%%%%%%%%%%%%%%%%%%%%%%%%%%%%%%%%%%%%%%
\section{Description of Space-time Manifold}


How to describe space-time manifold?
\begin{itemize}
\item
Metric (with a set of local coordinates), connection (Christoffel symbols).
\item
Metric (in the form of tetrads), connection (Ricci rotation coefficients).
\item
1+3 covariantly defined variables.
\end{itemize}



\section{Description of Space-time Manifold - Coordinates}

\section{Description of Space-time Manifold - Tetrads}




\section{Description of Space-time Manifold - 1+3 Covariant Description}

Physics in description is easier to understand.

\subsection{Definations}

Definations of some physical quantities and operators are listed below.

Here we have
\begin{enumerate}
	\item 
{\bf geometrical variables}: Volume
\item
{\bf Kinematical variables}: Velocity, Expansion rate, Shear rate
{\bf Thermaldynanmical variables}: Energy density, Momentum density, Pressure, Equation of state
\end{enumerate}




\paragraph{Volume}

To calculate volume, the volume element should be defined first in order to integrate. Before that, orientation on manifolds is to be figured out.

On an oriented manifold with metric, the defined volume element (a n-form) should be compatible with the orientation and also determined by the metric. \footnote{For more information, check out LIANG Canbin's book. Volume 1, page 115.}

Introducing those requirements, a compatible volume element is
\begin{equation}
\epsilon_{a_1\cdots a_n} = \pm \sqrt{|g|} (e^1)_{a_1}\wedge \cdots \wedge (e^n)_{a_n}
\end{equation}

Alternatively, this can be expressed in the way Ellis used in arXiv:gr-qc/9812046v5.
\begin{equation}
\eta_{abcd} = \eta_{[abcd]}, \quad \text{with} \eta_{0123} = \sqrt{|\mathrm {det} g_{ab}|}
\end{equation}

Induced volume element $\hat \epsilon_{a_1\cdots a_{n-1}}$ is defined use the normal vector $u^a$ of the hypersurface,
\begin{equation}
\hat \epsilon_{a_1\cdots a_{n-1}} = u^b \epsilon_{b a_1 \cdots a_{n-1}}
\end{equation}





\paragraph{4-velocity}
4-velocity of observed matter is 
\[u^\alpha = \frac{\mathrm d x^\alpha}{\mathrm d \tau}\]
with $u^\alpha u_\alpha =-1$, $\tau$ is the proper time along the worldlines of investaged matter.

\paragraph{Projection Tensors}

We can use 4-velocity to project variables to parts that is parallel to $u^\alpha$ and parts that is orthogonal to $u^\alpha$.

\begin{eqnarray}
U^a_{\phantom a b} &=& -u^a u_b \\
h_{ab} &=& g_{ab} + u_a u_b, \qquad \text{induced metric from $g_{ab}$}
\end{eqnarray}

Some properties of the  two projections.

\begin{eqnarray}
&& U^a_{\phantom a b} U^b_{\phantom bc} = U^a_{\phantom a c}  ,  U^a_{\phantom a a} = 1  , U_{ab}=g_{ac} U^c_{\phantom cb}  , U_{ab} u^b = - g_{ac} u^c u_b u^b = u_a \\
&& h^a_{\phantom ab} = g^{ac} h_{cb} = \delta^a_{\phantom ab} + u^a u_b = \delta^a_{\phantom ab} - U^a_{\phantom ab} \\
&& h^a_{\phantom a c}h^c_{\phantom c b} = (\delta^a_{\phantom ac} - U^a_{\phantom ac})(\delta^c_{\phantom cb} - U^c_{\phantom cb}) = \delta^a_{\phantom ab} - U^a_{\phantom ab} = h^a_{\phantom ab} \\
&& h^a_{\phantom aa} = 4-1 = 3  ,   h_{ab}u^b = 0
\end{eqnarray}





\paragraph{Covariant time derivative ($\dot \quad$)} This is the derivative along the fundamental worldlines (projection on the worldlines),

\begin{equation}
\dot T^{ab}_{\phantom{ab}cd} = u^e \nabla_e T^{ab}_{\phantom{ab}cd}
\end{equation}


\paragraph{Fully orthogonally projected covariant derivative ($\tilde \nabla$)}  This derivative is the project orghogonal to the normal vector of the hyperspace or orthogonal to the observer's 4-velocity or along the tagent of the hyperspace.

\begin{equation}
	\tilde\nabla_e T^{ab}_{\phantom{ab}cd} = h^a_f h^b_gh^p_ch^q_dh^r_e \nabla_r T^{fg}_{\phantom{fg}pq}
\end{equation}

\paragraph{Orthogonal projections of vectors}
	Orthogonal projection of vectors
\begin{equation}
v^{<a>}	= h^a_{\phantom a b} v^b
\end{equation}
 And the orthogonally projected symmetric trace-free part of tensors
 \begin{equation}
	T^{<ab>} = [h^{(a}_{\phantom {(a} c} h^{b)}_{\phantom{b)}d} - \frac{1}{3} h^{ab} h_{cd} ] T^{cd}
\end{equation}



\paragraph{Othogonal projected covariant time derivatives along $u^a$}
\begin{equation}
	\dot v^{<a>} = h^a_{\phantom a b} \dot v^b
\end{equation}
\begin{equation}
	\dot T^{<ab>} = [ h^{(a}_{\phantom{(a}b} h^{b)}_{\phantom{b)} d} - \frac 1 3 h^{ab}h_{cd} ]\dot T^{cd}
\end{equation}


\subsection{Properties}

\begin{itemize}
	\item 
Projected time and space derivatives of $U_{ab}$, $h_{ab}$ and $\eta_{abc}$ vanish.
\end{itemize}



















%%%%%%%%%%%%%%%%%%%%%%%%%%%%%%%%%%%%%%%%%%%%%
\section{Fields and Particles}

\subsection{Energy-Momentum Tensor for Particles}

\begin{equation}
S_p \equiv -m c \int \int \mathrm d s\mathrm d\tau \sqrt{-\dot x ^\mu g_{\mu\nu} \dot x^\nu} \delta^4(x^\mu - x^\mu (s))    ,
\end{equation}
in which $x^\mu(s)$ is the trajectory of the particle. Then the energy density $\rho$ corresponds to $m\delta^4(x^\mu- x^\mu(s))$.

The Largrange density
\begin{equation}
\mathcal L = -\int\mathrm ds mc \sqrt{-\dot x^\mu g_{\mu\nu}\dot x^\nu}\delta^4(x^\mu - x^\mu(s))
\end{equation}

Energy-momentum density is $\mathcal T^{\mu\nu} = \sqrt{-g}T^{\mu\nu}$ is
\begin{equation} 
\mathcal T^{\mu\nu} = -2 \frac{\partial \mathcal L}{\partial g_{\mu\nu}}
\end{equation}

Finally,
\begin{eqnarray}
\mathcal T^{\mu\nu} &=& \int \mathrm ds \frac{mc\dot x^\mu \dot x^\nu}{\sqrt{-\dot x^\mu g_{\mu\nu} \dot x^\nu}} \delta(t-t(s))\delta^3(\vec x - \vec x(t)) \\
&=& m\dot x^\mu \dot x^\nu \frac{\mathrm d s}{\mathrm d t} \delta^3(\vec x - \vec x(s(t)))
\end{eqnarray}





%%%%%%%%%%%%%%%%%%%%%%%%%%%%%%%%%%%%%%%%%%%%%%%%%%%%%%%%%%%%
\section{Theorems}

\subsection{Killing Vector Related}

\newtheorem{theorem}{}[chapter]

\begin{theorem}[]
$\xi^a$ is Killing vector field, $T^a$ is the tangent vector of geodesic line. Then $T^a\nabla_a(T^b\xi_b)=0$, that is $T^b\xi_b$ is a constant on geodesics.
\end{theorem}





%%%%%%%%%%%%%%%%%%%%%%%%%%%%%%%%%%%%%%%%%%%%%%%%%%%%%%%
\section{Specific Topics}
\subsection{Redshift}

In geometrical optics limit, the angular frequency $\omega$ of a photon with a 4-vector $K^a$, measured by a observer with a 4-velocity $Z^a$, is $\omega=-K_aZ^a$.

\subsection{Stationary vs Static}

\paragraph{Stationay}
"A stationary spacetime admits a timelike Killing vector field. That a stationary spacetime is one in which you can find a family of observers who observe no changes in the gravitational field (or sources such as matter or electromagnetic fields) over time."

When we say a field is stationary, we only mean the field is time-independent.

\paragraph{Static}
"A static spacetime is a stationary spacetime in which the timelike Killing vector field has vanishing vorticity, or equivalently (by the Frobenius theorem) is hypersurface orthogonal. A static spacetime is one which admits a slicing into spacelike hypersurfaces which are everywhere orthogonal to the world lines of our 'bored observers'"

When we say a field is static, the field is both time-independent and symmetric in a time reversal process.








\end{document}